\documentclass[12pt,a4paper]{article}

% ustawienia marginesu
\usepackage[left=1.6in,right=.8in,top=1.5in,bottom=1.5in]{geometry}

% polskie reguły dzielenia wyrazów itd
\usepackage{polski}

% polskie znaki zakodowane w UTF8
\usepackage[utf8]{inputenc}

% automatyczne generowanie odnośników w plikach PDF
\usepackage[pdftex,linkbordercolor={0 0.9 1}]{hyperref}

% pakiety matematyczne
\usepackage{amsthm,amsmath,amsfonts,amssymb,mathrsfs}

% ładne składanie odnośników do stron www
\usepackage{url}

% rozbudowane możliwości wypunktowań
\usepackage{enumerate}

% możliwość dodawania plików graficznych
\usepackage{graphicx}

%%% definicje twierdzeń itd :)
\newtheorem{tw}{Twierdzenie}[section]
\newtheorem{stw}[tw]{Stwierdzenie}
\newtheorem{fakt}[tw]{Fakt}
\newtheorem{lemat}[tw]{Lemat}

\theoremstyle{definition}
\newtheorem{df}[tw]{Definicja}
\newtheorem{ex}[tw]{Przykład}
\newtheorem{uw}[tw]{Uwaga}
\newtheorem{wn}[tw]{Wniosek}
\newtheorem{zad}{Zadanie}

% oznaczenia zbiorów liczbowych
\DeclareMathOperator{\R}{\mathbb{R}}
\DeclareMathOperator{\Z}{\mathbb{Z}}
\DeclareMathOperator{\N}{\mathbb{N}}
\DeclareMathOperator{\Q}{\mathbb{Q}}


% wartość bezwzględna, norma, iloczyn skalarny, nośnik, rozpięcie przestrzeni...
\providecommand{\abs}[1]{\left\lvert#1\right\rvert}
\providecommand{\var}[1]{\operatorname{var}(#1)}

% fajne nagłówki i stopki na stronie
\usepackage{fancyhdr}
\pagestyle{fancy}
\fancyhf{}
\fancyfoot[R]{\textbf{\thepage}}
\fancyhead[L]{\small\sffamily \nouppercase{\leftmark}}
\renewcommand{\headrulewidth}{0.4pt}
\renewcommand{\footrulewidth}{0.4pt}

% typowe dane dokumentu
\title{Funkcje ciągłe i różniczkowalne}
\date{\today}

% tu podaj swoje imię i nazwisko!
\author{Magdalena Bonczkowska}

% zaczynamy dokument
\begin{document}
% pokaż tytuł
\maketitle

% spis treści
\def\contentsname{Spis treści}
\tableofcontents

\section{Funkcje ciągłe}
\begin{df}
(funkcja ciągła). Niech $\ f:(a,b)\rightarrow\R$ oraz niech $ \ x_0\in(a,b)$. Mówimy, że funkcja $ \f$ jest ciągła w punkcie $ \x_0$ wtedy i tylko wtedy, gdy:
\[\forall_\varepsilon_>_0\exists_\delta_>_0\forall x\in(a,b)\mid\ x - x_o\mid <\delta\Rightarrow\mid f(x)-f(x_o)\mid<\varepsilon\]
\end{df}
\begin{ex}Wielomiany, funkcje trygonometryczne, wykładnicze, logarytmiczne
sa ciagłe w kazdym punkcie swojej dziedziny.
\end{ex}
\begin{ex}Funkcja $\ f$ dana wzorem:
$$f(x)=\left\{\begin{array}{cc}x \mbox { dla } x\neq 0 \\ 0\mbox{ dla }x=0\end{array}\right.$$
\\Jest ciągła w każdym punkcie poza $\ x_o=0$.
\\Niech  $\Q$ oznacza zbiór wszystkich liczb wymiernych. 
\end{ex}
\begin{ex}Funkcja $\ f$ dana wzorem:
$$f(x)=\left\{\begin{array}{cc}0 \mbox { dla } x\in \Q \\ 1\mbox{ dla }x\notin\Q\end{array}\right$$
nie jest ciągła w każdym punkcie.

\end{ex} 
\begin{ex}Funkcja $\ f$ dana wzorem:
$$f(x)=\left\{\begin{array}{cc}0 \mbox { dla } x\in \Q \\ x\mbox{ dla }x\notin\Q\end{array}\right$$
jest ciągła w punkcie $x_o=0$, ale nie jest ciągła w pozostałych punktach dzidziny.
\end{ex}
\begin{zad}
Udowodnij prawdziwość podanych przykładów. 
\end{zad}
\begin{df} Jeśli funkcja $\ f: A\rightarrow\R$ jest ciągła w każdym punkcie swojej dzidziny $ A$ to mówimy krótko, że jest ciągła.
\\Ponizsze twierdzenie zbiera podstawowe własnosci zbioru funkcji ciagłych.
\end{df}
\begin{tw} $ \ Niech\,funkcje  f,g: R$ $ \rightarrow\R$ $ \ będą\,ciągłe,\,oraz\,niech$ $ \alpha,\beta\in\R.$ $ \ Wtedy funkcje: $
\\a) h_1(x)=\alpha * f(x)+ \beta * g(x),
\\b) h_2(x)=f(x) * g(x),
\\c) h_3(x)=  $ \frac{f(x)}{g(x)} $ (o ile g(x) $\neq$ 0 dla dowolnego x \in \R),
\\d) h_4(x)=f(g(x)),  
\\$ są ciągłe. $

Następne twierdzenie zwane powszechnie „własnoscia Darboux” lub twierdzeniem
o wartosci posredniej ma liczne praktyczne zastosowania. Mówi ono o tym,
ze jesli funkcja ciagła przyjmuje jakies dwie wartosci, to przy odpowiednich załozeniach
co do dziedziny, przyjmuje tez wszystkie wartosci posrednie. Mozemy sobie to
łatwo wyobrazic na przykładzie funkcji, która opisuje zmiane temperatury w czasie. Jeśli o 7:00 było $-1^{\circ}$ C a o 9:00 było $2^{\circ}$ C. To zapewne gdzieś między 7:00, a 9:00 był taki moment, że temperatura wynosiła dokładnie $0^{\circ}$ C.


\end{tw}
\section{Różniczkowalność}
\begin{df}
Niech $\ f : (a,b)\rightarrow \R,x_0 \in(a,b)$ oraz $\ f$ ciągła w otoczeniu punktu $x_0$. Jeśli istnieje granica:
$$\lim_{x\rightarrow x0}\frac{f(x)-f(x_0)}{x-x_0}$$
i jest skończona, to oznaczamy przez $\ f'(x_0)$ i nazywamy pochodną funkcji $\ f$ w punkcie x_0.
\end{df}
\begin{df}
Jeśli funkcja $\ f$ posiada pochodną w każdym punkcie swojej dziediny,to mówimy, że $\ f$ jest różniczkowalna. Istnieje wtedy funkcja $\ f'$, która każdemu punktowi z dziediny funkcji $\ f$ przyporządkowuje warość pochodnej funkcji $\ f$ w tym punkcie.
\end{df}
\begin{ex}
Wielomiany, funkcje trygonometryczne, wykładnicze, logarytmiczne są różniczkowalne w każdym punkcie dziedziny.
\end{ex}
\begin{ex}
Funkcja $\ f(x) =\mid x \mid$ jest ciągła, ale nie posiada pochodnej w punkcie $ x_0=0$
\end{ex}
\begin{tw}
Niech $\ f:[a,b]\rightarrow\R ciągła\, i \,różniczkowalna\, na\, (a,b). Dodatkowo\, niech f'(x)\equiv 0\, dla\, x \in (a,b),\, oraz\, niech\, m=min_{x\in[a,b]}\,f(x),\, M = max_{x\in[a,b]}\,f(x). Wtedy\, na \,pewno f(a) = m,\, f(b) = M\, lub\, f(a) = M\, i\, f(b) = m.$ 
\end{tw}  
\end{document}
